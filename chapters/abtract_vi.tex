\begin{center}
\textbf{\large{Tóm tắt}	}
\end{center}

\addcontentsline{toc}{chapter}{Tóm tắt}

\begin{small}
Thực khuẩn là các vi-rút có khả năng xâm nhiễm vào vi khuẩn. Dựa trên chu kỳ sống, thực khuẩn được phân thành hai nhóm chính. Nhóm thứ nhất là thực khuẩn thể độc lực, thực hiện chu kỳ tan, trong đó thực khuẩn nhân lên nhanh chóng và phá hủy tế bào vi khuẩn sau khi xâm nhiễm. Nhóm thứ hai là thực khuẩn thể ôn hòa, có khả năng tích hợp bộ gen của mình vào nhiễm sắc thể của vi khuẩn và sao chép cùng tế bào chủ thông qua chu kỳ tiềm tan. Từ đó, việc xác định chính xác chu kỳ sống của thực khuẩn là một bước quan trọng trong việc phát triển các ứng dụng phù hợp, đặc biệt trong trị liệu bằng phage, một lựa chọn tiềm năng cho các bệnh nhân dị ứng hoặc kháng thuốc kháng sinh.

Trước đây, các phương pháp phân loại thực khuẩn truyền thống chủ yếu dựa trên nuôi cấy trong phòng thí nghiệm, vốn đòi hỏi nhiều thời gian, chi phí cao và không hiệu quả khi xử lý khối lượng lớn dữ liệu chưa được gán nhãn. Hiện nay, dữ liệu di truyền ngày càng phong phú, được thu thập từ các nguồn như NCBI, PhageScope hay PhageDB cho phép thực hiện các phương pháp phân loại phage dựa trên tính toán. Với sự phát triển của công nghệ, các kỹ thuật học máy và học sâu đã được áp dụng để thực hiện phân loại thực khuẩn đạt được các kết quả tốt và rút ngắn thời gian phân loại so với các phương pháp truyền thống.

Mục tiêu của báo cáo là cung cấp một cái nhìn tổng quan và hệ thống về bài toán phân loại thực khuẩn dựa trên phương pháp tính toán. Trong báo cáo này, nhóm báo cáo chia các phương pháp thành hai nhóm chính: (1) các phương pháp học máy truyền thống giải quyết bài toán với dữ liệu bộ gen đầy đủ (PHACTS, PhageAI, BACPHLIP), và (2) các phương pháp học sâu có khả năng xử lý bài toán với dữ liệu không hoàn chỉnh (DeePhage, PhaTYP, DeepPL).

Báo cáo gồm có 4 chương:
\begin{enumerate}
    \item \textbf{Giới thiệu.} Nội dung của chương này là giới thiệu về các khái niệm liên quan, phát biểu bài toán và nguồn dữ liệu.
    \item \textbf{Các nghiên cứu liên quan.} Chương này tập trung trình bày về các phương pháp phân loại thực khuẩn dựa trên tính toán tiêu biểu.
    \item \textbf{Thực nghiệm.} Chương này trình bày về dữ liệu, kịch bản, chỉ số, kết quả liên quan tới các thực nghiệm mà nhóm báo cáo đạt được.
    \item \textbf{Kết luận.} Trong chương này, nhóm báo cáo tập trung thảo luận về kết quả đạt được và hướng nghiên cứu tiếp theo.
\end{enumerate}
% \vspace*{1cm}
% \textbf{Từ khóa}: \textit{hệ thống dòng sản phẩm}, \textit{kiểm thử phần mềm}, \textit{sinh ca kiểm thử}, \textit{kiểm thử đột biến}, \textit{dữ liệu đánh giá}

\end{small}
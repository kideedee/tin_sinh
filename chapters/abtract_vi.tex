\begin{center}
\textbf{\large{Tóm tắt}	}
\end{center}

\addcontentsline{toc}{chapter}{Tóm tắt}

\begin{small}
Thực khuẩn (bacteriophage, gọi tắt là phage) là các virus có khả năng xâm nhiễm vào vi khuẩn. Dựa trên chu kỳ sống, thực khuẩn được phân thành hai nhóm chính. Nhóm thứ nhất là thực khuẩn thể độc lực (virulent phages hay lytic phages), thực hiện chu kỳ tan (lytic cycle), trong đó thực khuẩn nhân lên nhanh chóng và phá hủy tế bào vi khuẩn sau khi xâm nhiễm. Nhóm thứ hai là thực khuẩn thể ôn hòa (temperate phages hay lysogenic phages), có khả năng tích hợp bộ gen của mình vào nhiễm sắc thể của vi khuẩn và sao chép cùng tế bào chủ thông qua chu kỳ tiềm tan (lysogenic cycle). Từ đó, việc xác định chính xác chu kỳ sống của thực khuẩn là một bước quan trọng trong việc phát triển các ứng dụng phù hợp, đặc biệt trong trị liệu bằng phage – một lựa chọn tiềm năng cho các bệnh nhân dị ứng hoặc kháng thuốc kháng sinh.

Trước đây, các phương pháp phân loại thực khuẩn truyền thống chủ yếu dựa trên nuôi cấy trong phòng thí nghiệm, vốn đòi hỏi nhiều thời gian, chi phí cao và không hiệu quả khi xử lý khối lượng lớn dữ liệu chưa được gán nhãn. Hiện nay, dữ liệu di truyền ngày càng phong phú, được thu thập từ các nguồn như NCBI, PhageScope hay PhageDB cho phép thực hiện các phương pháp phân loại phage dựa trên tính toán. Theo trình tự thời gian, các kỹ thuật học máy (machine learning) và học sâu (deep learning) đã được áp dụng để thực hiện phân loại thực khuẩn để nâng cao hiệu quả, độ chính xác cũng như đáp ứng được với khả năng xử lí dữ liệu không hoàn chỉnh.

Báo cáo này tập trung khảo sát các phương pháp học phân loại thực khuẩn, được chia thành hai nhóm chính: (1) các phương pháp học máy truyền thống sử dụng bộ gen đầy đủ (bao gồm PHACTS, PhageAI, BACPHLIP), và (2) các phương pháp học sâu có khả năng xử lý dữ liệu không hoàn chỉnh (bao gồm DeePhage, PhaTYP, DeepPL). Các phương pháp này được so sánh dựa trên kỹ thuật sử dụng, độ chính xác, khả năng mở rộng và tiềm năng ứng dụng thực tiễn.

Mục tiêu của báo cáo là cung cấp một cái nhìn tổng quan và hệ thống về hiện trạng nghiên cứu trong lĩnh vực phân loại thực khuẩn, các thành tựu đã đạt được ở thời điểm đầu năm 2025, đồng thời đánh giá tiềm năng phát triển của các hướng tiếp cận hiện nay.

% \vspace*{1cm}
% \textbf{Từ khóa}: \textit{hệ thống dòng sản phẩm}, \textit{kiểm thử phần mềm}, \textit{sinh ca kiểm thử}, \textit{kiểm thử đột biến}, \textit{dữ liệu đánh giá}

\end{small}